\chapter{Cơ sở lý thuyết}

\section{Lý do chọn AES‑256‑GCM}
\textbf{AES} là chuẩn mực công nghiệp, được tăng tốc phần cứng và hỗ trợ native bởi Web Crypto API. \textbf{GCM} cung cấp cả bảo mật \textit{bí mật} lẫn \textit{toàn vẹn/xác thực} (AEAD). Kích thước IV 96‑bit tiêu chuẩn giúp hiệu năng tốt và đơn giản khi triển khai. Với khoá 256‑bit, không có tấn công thực tế tốt hơn vét cạn.

\section{Tổng quan AES}
AES là mã khối kích thước 128‑bit, cấu trúc SPN với các bước \textit{SubBytes}, \textit{ShiftRows}, \textit{MixColumns}, \textit{AddRoundKey}. Số vòng phụ thuộc độ dài khoá (14 vòng cho 256‑bit).

\begin{figure}[H]
  \centering
  \begin{tikzpicture}[node distance=1.2cm]
    \node (pt) [block] {Plaintext 128‑bit};
    \node (rk0) [smallblock, below=of pt] {AddRoundKey};
    \node (round) [smallblock, below=of rk0] {SubBytes → ShiftRows → MixColumns → AddRoundKey};
    \node (dots) [below=0.2cm of round] {$\times$ (Nr−1) vòng};
    \node (final) [smallblock, below=0.2cm of dots] {SubBytes → ShiftRows → AddRoundKey};
    \node (ct) [block, below=of final] {Ciphertext 128‑bit};
    \draw[arrow] (pt) -- (rk0);
    \draw[arrow] (rk0) -- (round);
    \draw[arrow] (round) -- (dots);
    \draw[arrow] (dots) -- (final);
    \draw[arrow] (final) -- (ct);
  \end{tikzpicture}
  \caption{Cấu trúc vòng AES (minh hoạ)}
\end{figure}

\section{GCM và tính xác thực}
GCM kết hợp CTR để mã hóa và GHASH để tính nhãn xác thực (tag). Đầu vào gồm khoá bí mật, IV (không lặp), dữ liệu kèm theo AAD (tuỳ chọn), bản rõ; đầu ra gồm bản mã và tag xác thực. Sai IV hoặc thay đổi dữ liệu khiến giải mã thất bại.

\section{PBKDF2 và bao bọc khoá}
Khi người dùng chọn passphrase, hệ thống dùng PBKDF2‑HMAC‑SHA‑256 (200k vòng) sinh khoá dẫn xuất để \textbf{bao bọc khoá AES} bằng AES‑GCM (key wrapping). Lưu trữ: salt (16B), ivWrap (12B) và wrappedKey. Nếu ở chế độ demo, có thể lưu khoá dạng base64 (không khuyến nghị sản xuất).

